\documentclass[a4paper,12pt]{article}

\usepackage{circuitikz}

\usepackage{amsmath}

\begin{document}

    There can be a time delay before an error and a failure, or a failure may never even occur from an error. Same with going from faults to errors.

    On example of liveware faults was pilots who required training after controls became digital.

    \begin{itemize}
        \item Permanent errors must be repaired
        \item Intermittent errors depend on the error: Do we fix the error or just modify the cases which cause the error?
        \item Transient errors cannot be easily fixed. They need to be tolerated.
    \end{itemize}

    Consider the circuit $f = \overline{A + B}$

    \begin{circuitikz}[baseline=(current bounding box.north)]
        \node (A) at (0, 2) {A};
        \node (B) at (0, 0) {B};
        \node[xor port] (C) at (2, 1) {};
        \draw (A) -| (C.in 1);
        \draw (B) -| (C.in 2);
    \end{circuitikz}
    \parbox[t]{10cm}{\vskip0pt
        To test a fault, we need to set the faulty input to the complement and the other inputs to be non-dominating. So to test A-sa0, we need $A = 1, B = 0$ and will get 1 if the fault isn't there, or 1 if the fault is. Likewise, to test A-sa1, we need $A = 0, B = 0$ and will get 0 if the fault isn't there, and 1 if the fault is.
    }

    Note that:
    \begin{itemize}
        \item $<0, 0>$ will detect A-sa1, B-sa1 and f-sa0.
        \item $<0, 1>$ will detect B-sa0 and f-sa1.
        \item $<1, 0>$ will detect A-sa0 and f-sa1.
    \end{itemize}

    So our test set is $\{00, 01, 10\}$

    Boolean difference example (circuit in slides):

    To detect y-sa0, find values such that $y\frac{df}{dy} = 1$:

    $y(f(y = 0) \oplus f(y = 1)) = 1 \implies y(z \oplus x) = 1 \implies \overline{x}yz + xy\overline{z} = 1$

    Therefore, 011 and 110 can detect the fault.

\end{document}
